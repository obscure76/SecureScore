
%% bare_conf.tex
%% V1.3
%% 2007/01/11
%% by Michael Shell
%% See:
%% http://www.michaelshell.org/
%% for current contact information.
%%
%% This is a skeleton file demonstrating the use of IEEEtran.cls
%% (requires IEEEtran.cls version 1.7 or later) with an IEEE conference paper.
%%
%% Support sites:
%% http://www.michaelshell.org/tex/ieeetran/
%% http://www.ctan.org/tex-archive/macros/latex/contrib/IEEEtran/
%% and
%% http://www.ieee.org/

%%*************************************************************************
%% Legal Notice:
%% This code is offered as-is without any warranty either expressed or
%% implied; without even the implied warranty of MERCHANTABILITY or
%% FITNESS FOR A PARTICULAR PURPOSE! 
%% User assumes all risk.
%% In no event shall IEEE or any contributor to this code be liable for
%% any damages or losses, including, but not limited to, incidental,
%% consequential, or any other damages, resulting from the use or misuse
%% of any information contained here.
%%
%% All comments are the opinions of their respective authors and are not
%% necessarily endorsed by the IEEE.
%%
%% This work is distributed under the LaTeX Project Public License (LPPL)
%% ( http://www.latex-project.org/ ) version 1.3, and may be freely used,
%% distributed and modified. A copy of the LPPL, version 1.3, is included
%% in the base LaTeX documentation of all distributions of LaTeX released
%% 2003/12/01 or later.
%% Retain all contribution notices and credits.
%% ** Modified files should be clearly indicated as such, including  **
%% ** renaming them and changing author support contact information. **
%%
%% File list of work: IEEEtran.cls, IEEEtran_HOWTO.pdf, bare_adv.tex,
%%                    bare_conf.tex, bare_jrnl.tex, bare_jrnl_compsoc.tex
%%*************************************************************************

% *** Authors should verify (and, if needed, correct) their LaTeX system  ***
% *** with the testflow diagnostic prior to trusting their LaTeX platform ***
% *** with production work. IEEE's font choices can trigger bugs that do  ***
% *** not appear when using other class files.                            ***
% The testflow support page is at:
% http://www.michaelshell.org/tex/testflow/



% Note that the a4paper option is mainly intended so that authors in
% countries using A4 can easily print to A4 and see how their papers will
% look in print - the typesetting of the document will not typically be
% affected with changes in paper size (but the bottom and side margins will).
% Use the testflow package mentioned above to verify correct handling of
% both paper sizes by the user's LaTeX system.
%
% Also note that the "draftcls" or "draftclsnofoot", not "draft", option
% should be used if it is desired that the figures are to be displayed in
% draft mode.
%
\documentclass[conference]{IEEEtran}
% Add the compsoc option for Computer Society conferences.
%
% If IEEEtran.cls has not been installed into the LaTeX system files,
% manually specify the path to it like:
% \documentclass[conference]{../sty/IEEEtran}



\setcounter{secnumdepth}{4}

% Some very useful LaTeX packages include:
% (uncomment the ones you want to load)


% *** MISC UTILITY PACKAGES ***
%
%\usepackage{ifpdf}
% Heiko Oberdiek's ifpdf.sty is very useful if you need conditional
% compilation based on whether the output is pdf or dvi.
% usage:
% \ifpdf
%   % pdf code
% \else
%   % dvi code
% \fi
% The latest version of ifpdf.sty can be obtained from:
% http://www.ctan.org/tex-archive/macros/latex/contrib/oberdiek/
% Also, note that IEEEtran.cls V1.7 and later provides a builtin
% \ifCLASSINFOpdf conditional that works the same way.
% When switching from latex to pdflatex and vice-versa, the compiler may
% have to be run twice to clear warning/error messages.






% *** CITATION PACKAGES ***
%
%\usepackage{cite}
% cite.sty was written by Donald Arseneau
% V1.6 and later of IEEEtran pre-defines the format of the cite.sty package
% \cite{} output to follow that of IEEE. Loading the cite package will
% result in citation numbers being automatically sorted and properly
% "compressed/ranged". e.g., [1], [9], [2], [7], [5], [6] without using
% cite.sty will become [1], [2], [5]--[7], [9] using cite.sty. cite.sty's
% \cite will automatically add leading space, if needed. Use cite.sty's
% noadjust option (cite.sty V3.8 and later) if you want to turn this off.
% cite.sty is already installed on most LaTeX systems. Be sure and use
% version 4.0 (2003-05-27) and later if using hyperref.sty. cite.sty does
% not currently provide for hyperlinked citations.
% The latest version can be obtained at:
% http://www.ctan.org/tex-archive/macros/latex/contrib/cite/
% The documentation is contained in the cite.sty file itself.






% *** GRAPHICS RELATED PACKAGES ***
%
\ifCLASSINFOpdf
  % \usepackage[pdftex]{graphicx}
  % declare the path(s) where your graphic files are
  % \graphicspath{{../pdf/}{../jpeg/}}
  % and their extensions so you won't have to specify these with
  % every instance of \includegraphics
  % \DeclareGraphicsExtensions{.pdf,.jpeg,.png}
\else
  % or other class option (dvipsone, dvipdf, if not using dvips). graphicx
  % will default to the driver specified in the system graphics.cfg if no
  % driver is specified.
  % \usepackage[dvips]{graphicx}
  % declare the path(s) where your graphic files are
  % \graphicspath{{../eps/}}
  % and their extensions so you won't have to specify these with
  % every instance of \includegraphics
  % \DeclareGraphicsExtensions{.eps}
\fi
% graphicx was written by David Carlisle and Sebastian Rahtz. It is
% required if you want graphics, photos, etc. graphicx.sty is already
% installed on most LaTeX systems. The latest version and documentation can
% be obtained at: 
% http://www.ctan.org/tex-archive/macros/latex/required/graphics/
% Another good source of documentation is "Using Imported Graphics in
% LaTeX2e" by Keith Reckdahl which can be found as epslatex.ps or
% epslatex.pdf at: http://www.ctan.org/tex-archive/info/
%
% latex, and pdflatex in dvi mode, support graphics in encapsulated
% postscript (.eps) format. pdflatex in pdf mode supports graphics
% in .pdf, .jpeg, .png and .mps (metapost) formats. Users should ensure
% that all non-photo figures use a vector format (.eps, .pdf, .mps) and
% not a bitmapped formats (.jpeg, .png). IEEE frowns on bitmapped formats
% which can result in "jaggedy"/blurry rendering of lines and letters as
% well as large increases in file sizes.
%
% You can find documentation about the pdfTeX application at:
% http://www.tug.org/applications/pdftex





% *** MATH PACKAGES ***
%
%\usepackage[cmex10]{amsmath}
% A popular package from the American Mathematical Society that provides
% many useful and powerful commands for dealing with mathematics. If using
% it, be sure to load this package with the cmex10 option to ensure that
% only type 1 fonts will utilized at all point sizes. Without this option,
% it is possible that some math symbols, particularly those within
% footnotes, will be rendered in bitmap form which will result in a
% document that can not be IEEE Xplore compliant!
%
% Also, note that the amsmath package sets \interdisplaylinepenalty to 10000
% thus preventing page breaks from occurring within multiline equations. Use:
%\interdisplaylinepenalty=2500
% after loading amsmath to restore such page breaks as IEEEtran.cls normally
% does. amsmath.sty is already installed on most LaTeX systems. The latest
% version and documentation can be obtained at:
% http://www.ctan.org/tex-archive/macros/latex/required/amslatex/math/





% *** SPECIALIZED LIST PACKAGES ***
%
%\usepackage{algorithmic}
% algorithmic.sty was written by Peter Williams and Rogerio Brito.
% This package provides an algorithmic environment fo describing algorithms.
% You can use the algorithmic environment in-text or within a figure
% environment to provide for a floating algorithm. Do NOT use the algorithm
% floating environment provided by algorithm.sty (by the same authors) or
% algorithm2e.sty (by Christophe Fiorio) as IEEE does not use dedicated
% algorithm float types and packages that provide these will not provide
% correct IEEE style captions. The latest version and documentation of
% algorithmic.sty can be obtained at:
% http://www.ctan.org/tex-archive/macros/latex/contrib/algorithms/
% There is also a support site at:
% http://algorithms.berlios.de/index.html
% Also of interest may be the (relatively newer and more customizable)
% algorithmicx.sty package by Szasz Janos:
% http://www.ctan.org/tex-archive/macros/latex/contrib/algorithmicx/




% *** ALIGNMENT PACKAGES ***
%
%\usepackage{array}
% Frank Mittelbach's and David Carlisle's array.sty patches and improves
% the standard LaTeX2e array and tabular environments to provide better
% appearance and additional user controls. As the default LaTeX2e table
% generation code is lacking to the point of almost being broken with
% respect to the quality of the end results, all users are strongly
% advised to use an enhanced (at the very least that provided by array.sty)
% set of table tools. array.sty is already installed on most systems. The
% latest version and documentation can be obtained at:
% http://www.ctan.org/tex-archive/macros/latex/required/tools/


%\usepackage{mdwmath}
%\usepackage{mdwtab}
% Also highly recommended is Mark Wooding's extremely powerful MDW tools,
% especially mdwmath.sty and mdwtab.sty which are used to format equations
% and tables, respectively. The MDWtools set is already installed on most
% LaTeX systems. The lastest version and documentation is available at:
% http://www.ctan.org/tex-archive/macros/latex/contrib/mdwtools/


% IEEEtran contains the IEEEeqnarray family of commands that can be used to
% generate multiline equations as well as matrices, tables, etc., of high
% quality.


%\usepackage{eqparbox}
% Also of notable interest is Scott Pakin's eqparbox package for creating
% (automatically sized) equal width boxes - aka "natural width parboxes".
% Available at:
% http://www.ctan.org/tex-archive/macros/latex/contrib/eqparbox/





% *** SUBFIGURE PACKAGES ***
%\usepackage[tight,footnotesize]{subfigure}
% subfigure.sty was written by Steven Douglas Cochran. This package makes it
% easy to put subfigures in your figures. e.g., "Figure 1a and 1b". For IEEE
% work, it is a good idea to load it with the tight package option to reduce
% the amount of white space around the subfigures. subfigure.sty is already
% installed on most LaTeX systems. The latest version and documentation can
% be obtained at:
% http://www.ctan.org/tex-archive/obsolete/macros/latex/contrib/subfigure/
% subfigure.sty has been superceeded by subfig.sty.



%\usepackage[caption=false]{caption}
%\usepackage[font=footnotesize]{subfig}
% subfig.sty, also written by Steven Douglas Cochran, is the modern
% replacement for subfigure.sty. However, subfig.sty requires and
% automatically loads Axel Sommerfeldt's caption.sty which will override
% IEEEtran.cls handling of captions and this will result in nonIEEE style
% figure/table captions. To prevent this problem, be sure and preload
% caption.sty with its "caption=false" package option. This is will preserve
% IEEEtran.cls handing of captions. Version 1.3 (2005/06/28) and later 
% (recommended due to many improvements over 1.2) of subfig.sty supports
% the caption=false option directly:
%\usepackage[caption=false,font=footnotesize]{subfig}
%
% The latest version and documentation can be obtained at:
% http://www.ctan.org/tex-archive/macros/latex/contrib/subfig/
% The latest version and documentation of caption.sty can be obtained at:
% http://www.ctan.org/tex-archive/macros/latex/contrib/caption/




% *** FLOAT PACKAGES ***
%
%\usepackage{fixltx2e}
% fixltx2e, the successor to the earlier fix2col.sty, was written by
% Frank Mittelbach and David Carlisle. This package corrects a few problems
% in the LaTeX2e kernel, the most notable of which is that in current
% LaTeX2e releases, the ordering of single and double column floats is not
% guaranteed to be preserved. Thus, an unpatched LaTeX2e can allow a
% single column figure to be placed prior to an earlier double column
% figure. The latest version and documentation can be found at:
% http://www.ctan.org/tex-archive/macros/latex/base/



%\usepackage{stfloats}
% stfloats.sty was written by Sigitas Tolusis. This package gives LaTeX2e
% the ability to do double column floats at the bottom of the page as well
% as the top. (e.g., "\begin{figure*}[!b]" is not normally possible in
% LaTeX2e). It also provides a command:
%\fnbelowfloat
% to enable the placement of footnotes below bottom floats (the standard
% LaTeX2e kernel puts them above bottom floats). This is an invasive package
% which rewrites many portions of the LaTeX2e float routines. It may not work
% with other packages that modify the LaTeX2e float routines. The latest
% version and documentation can be obtained at:
% http://www.ctan.org/tex-archive/macros/latex/contrib/sttools/
% Documentation is contained in the stfloats.sty comments as well as in the
% presfull.pdf file. Do not use the stfloats baselinefloat ability as IEEE
% does not allow \baselineskip to stretch. Authors submitting work to the
% IEEE should note that IEEE rarely uses double column equations and
% that authors should try to avoid such use. Do not be tempted to use the
% cuted.sty or midfloat.sty packages (also by Sigitas Tolusis) as IEEE does
% not format its papers in such ways.





% *** PDF, URL AND HYPERLINK PACKAGES ***
%
%\usepackage{url}
% url.sty was written by Donald Arseneau. It provides better support for
% handling and breaking URLs. url.sty is already installed on most LaTeX
% systems. The latest version can be obtained at:
% http://www.ctan.org/tex-archive/macros/latex/contrib/misc/
% Read the url.sty source comments for usage information. Basically,
% \url{my_url_here}.





% *** Do not adjust lengths that control margins, column widths, etc. ***
% *** Do not use packages that alter fonts (such as pslatex).         ***
% There should be no need to do such things with IEEEtran.cls V1.6 and later.
% (Unless specifically asked to do so by the journal or conference you plan
% to submit to, of course. )


% correct bad hyphenation here
\hyphenation{op-tical net-works semi-conduc-tor}


\begin{document}
%
% paper title
% can use linebreaks \\ within to get better formatting as desired
\title{SDN attack and mitigation}

%\usepackage[utf8]{inputenc}
% author names and affiliations
% use a multiple column layout for up to three different
% affiliations
\author{
\IEEEauthorblockN{Midhun Prabhu Chinta}
\IEEEauthorblockA{UIN:722005998\\Department of Computer Science\\ and Engineering\\
Texas A\&M University\\
CollegeStation, Texas-77840\\
Email: midhun@cs.tamu.edu}
\and
\IEEEauthorblockN{Viswanath Boga}
\IEEEauthorblockA{UIN:722005998\\Department of Computer Science\\ and Engineering\\
Texas A\&M University\\
CollegeStation, Texas-77840\\
Email: midhun@cs.tamu.edu}
\and
\IEEEauthorblockN{Rajesh Kumar Nayak}
\IEEEauthorblockA{UIN:722005998\\Department of Computer Science\\ and Engineering\\
Texas A\&M University\\
CollegeStation, Texas-77840\\
Email: midhun@cs.tamu.edu}
}

% conference papers do not typically use \thanks and this command
% is locked out in conference mode. If really needed, such as for
% the acknowledgment of grants, issue a \IEEEoverridecommandlockouts
% after \documentclass

% for over three affiliations, or if they all won't fit within the width
% of the page, use this alternative format:
% 
%\author{\IEEEauthorblockN{Michael Shell\IEEEauthorrefmark{1},
%Homer Simpson\IEEEauthorrefmark{2},
%James Kirk\IEEEauthorrefmark{3}, 
%Montgomery Scott\IEEEauthorrefmark{3} and
%Eldon Tyrell\IEEEauthorrefmark{4}}
%\IEEEauthorblockA{\IEEEauthorrefmark{1}School of Electrical and Computer Engineering\\
%Georgia Institute of Technology,
%Atlanta, Georgia 30332--0250\\ Email: see http://www.michaelshell.org/contact.html}
%\IEEEauthorblockA{\IEEEauthorrefmark{2}Twentieth Century Fox, Springfield, USA\\
%Email: homer@thesimpsons.com}
%\IEEEauthorblockA{\IEEEauthorrefmark{3}Starfleet Academy, San Francisco, California 96678-2391\\
%Telephone: (800) 555--1212, Fax: (888) 555--1212}
%\IEEEauthorblockA{\IEEEauthorrefmark{4}Tyrell Inc., 123 Replicant Street, Los Angeles, California 90210--4321}}




% use for special paper notices
%\IEEEspecialpapernotice{(Invited Paper)}




% make the title area
\maketitle


\begin{abstract}
%\boldmath
OpenFlow(a switching technology) is getting popular day by day. OpenFlow Protocol has caught the interest of Network device manufacturers, vendors and administrators. This is because of its flexible nature in supporting different innovative protocols without getting to know the details implementation of the manufacturer code. Software-Defined Networking (SDN), a network that can be configured and managed from a centralized point called controller and used the OpenFlow technology to communicate between the controller and the network device. SDN has provided a new paradigm for providing flexibility and programmability to conventional networks. With shift of control plane from distributed to centralized design, the entire network is prone to attack at a single point of network. In addition to external attacks on the centralized Controller, there are some attacks that can be induced by malicious hosts lying inside the network. As part of our work, we explored some new attacks on the SDN like ARP (Address Resolution Protocol) broadcast flooding , ping packet with heavy payload , UDP packets with heavy payload and DDoS attacks (Amplification exploiting NTP, DNS servers) targeting any host or link connected to the target host. In order to make the defense adaptive in nature irrespective of protocol, we have used machine learning algorithms and statistics models to fit the data in the best possible way . Our results show that we are able to prevent the most of the attacks using the machine learning technique in the SDN to protect the targeted host.
\end{abstract}
% IEEEtran.cls defaults to using nonbold math in the Abstract.
% This preserves the distinction between vectors and scalars. However,
% if the conference you are submitting to favors bold math in the abstract,
% then you can use LaTeX's standard command \boldmath at the very start
% of the abstract to achieve this. Many IEEE journals/conferences frown on
% math in the abstract anyway.

% no keywords
\begin{IEEEkeywords}
SoftwareDefinedNetworking(SDN), ARP, ICMP, DOS, DDOS.
\end{IEEEkeywords}



% For peer review papers, you can put extra information on the cover
% page as needed:
% \ifCLASSOPTIONpeerreview
% \begin{center} \bfseries EDICS Category: 3-BBND \end{center}
% \fi
%
% For peerreview papers, this IEEEtran command inserts a page break and
% creates the second title. It will be ignored for other modes.
\IEEEpeerreviewmaketitle



\section{\textbf{Introduction}}
% no \IEEEPARstart
SDN or software defined network changes the computer network architecture by decoupling the data plane from the control plane. Data plane involves in the forwarding of packet to a particular/all interface and control plane acts as the decision maker or brain of the system. SDN provides a greater control of the operation of the network by allowing central and programmable control of the switches connected in the network. Controller has the flexibility in programming the network. This eventually opens up some security concerns as attackers now able to exploit the flexibility feature of the controller to attack the network. As controller is the single point of contact for the underlying network, any error at the controller or controller operation mode will be broadcasted to the entire network. Therefore it is crucial to secure the overall functionality of the controller so that it can’t be exploited by attackers to target the network underneath. Similarly as controller is the central point in the network, it is very much possible for it to detect any attacks in its network by collecting all the information from different switches connected in its network in a very efficient and robust way.
%\paragraph
%\setlength{\parindent}{1cm}
\par
Denial-of-service (DoS) attack is an attempt to make a machine or resource unavailable to its intended users. Distributed Denial of Service (DDoS) is type of DoS attack in which attackers attack the machine or resources in a distributed fashion. It means attackers used multiple machines to execute the attack against a particular machine. In this type of attack, attackers instruct a set of compromised machines called Bots to flood the intended victim with large number of request. In order to prevent such attacks network professionals use different techniques like throttling, rate limiting etc to detect abnormal traffics on a particular link or a
server. Attackers then used different other techniques like sending low intensity traffic not to the targeted machine rather than all the hosts connected with the same link with the target machine so that link will be flooded with traffic and target machine will be virtually cutoff from the network. In this type of attack, the attackers act as a bot master and use its bot army to send low intensity but constant flow of legitimate traffic on the specific link in the path to which the
server/machine is connected. In this way the link will be saturated to its maximum capacity and the target machine will be virtually disconnected from the network. This type of attacks are really difficult for victim to detect. But in SDN the controller has overall idea about the network and it can get all the network statistics from the different switch connected in its network. So it is possible for the controller to monitor traffic in its network and use machine learning techniques
to learn such patterns and diagnose such attacks in its network and prevent them by dropping such traffic for certain period of time or filtering out completely from those sources. In our work we have attacked the target host connected in the SDN by various attack techniques like DDoS attack, through flooding ARP broadcast packet to unknown host, ping packets with heavy payload by various malicious host (called bot army) to target host, UDP packets with heavy payloads to the target host with random destination port number. ARP broadcast packet to unknown host will be replicated to all the interfaces connected to that particular switch which will increase the network traffic unnecessarily.
% You must have at least 2 lines in the paragraph with the drop letter
% (should never be an issue)
%I wish you the best of success.

%\hfill mds
%\hfill January 11, 2007



\section{\textbf{Motivation}}




\section{\textbf{Related Work}}
In the traditional architecture of SDN, the controller has a REST interface to talk to different applications and offer services. However, this ability offers security vulnerability which was addressed through rosemary[1]. AvantGuard [2] secures the controller from external attacks which aim at exhausting CPU of controller with TCP SYN flood attack, malicious scans by hosts in the network and network intrusion detection. However, this does not cover all attacks possible within a SDN and we can exploit the programmability to the controller to security of the SDN as a whole. In [4], there was an entropy based implementation on detecting the anomaly in the SDN and accordingly take appropriate action to mitigate such attack. In [8], there is a discussion on most recent DDOS attacks in the real world network.



\section{\textbf{System Design}}




\section{\textbf{Implementation}}
The changes required for our project can be broadly divided into two categories- Changes at Controller, Changes at the external module.
\subsection{\textbf{Controller:}}
The controller on the flood light doesn’t get the opportunity to analyze the packets since the time a particular flow is pushed to the switch. The flow rule is matched to the IP addresses and forwarded to the hosts without any look up on the IPs. In fact the packet can have a NULL IP address(or a Spoofed one) as the source and set the destination IP as the target. The switch wouldn’t or shouldn’t bother the IP layer. However, controller have the authority to clean the flows on the switch at any time and dump the flows it needs for the purpose of detection. We need to know when a particular host/switch has started to behave badly. For this, we capture the packets on a switch per flow. A period can be set with a certain time or number of packets. We take the approach of number of packets for a flow as soon as the flow is pushed to a switch. The flow pushed is set with the action to forward each packet to the controller. The action OFPP\_Controller is used for this. First 10 packets are stored for the analysis. After the first 10 packets, we will flush the particular flow F and push a fresh flow F’ with action forward to controller removed. Note that during this time, there may be some packets reaching the controller causing flow loss. These needs to be forwarded to the appropriate switch. However, this behavior is similar to the IDLEtimeout of a flow on the switch.
\subsection{\textbf{External module:}}
All the algorithms mentioned in section 2 are implemented as part of external python module. The scikit machine learning library in python is used for KMeans clustering and cross-validation schemes. The external module talks with controller on a TCP connection. The python module collects packets regularly, analyzes them and updates the trust scores back to the controller. Based on trust score, appropriate action is taken by the controller. For the effective communication process, all the information to the module is send as a json object and so is the information that is sent back to the controller. Once the algorithms are trained, it will detect a packet malign or benign. So, detection for any of the above mentioned attacks accumulates the malign packets over time. So, if we see high percentage of malign packets for consequent runs, we detect that an attack is on its way in the network.
\subsubsection{\textbf{Control flow attack analysis:}}
Control message saturation analysis-To find the malign packet contribution from each switch since we know the identity of each switch which sent a malign packet. We have updated the trust score for a switch depending on the contribution of that switch to malign packets.
\par
\textbf{Mitigation:}The control messages from a switch to controller are currently going through TCP connection. To limit a rogue switch (or a victim switch due to a malign host) to flood the controller , we limit the TCP receive window with respect to that switch for a random period of time.

\subsubsection{\textbf{Dataplane attack analysis}}
For each host which is essentially associated with a switchport, we propose a novel approach to assign an attackScore, victimScore to each host. We measure the Number of packets in, Amplification factor In and measure the standard deviation of both of them. As the standard deviation is more (based on the threshold we set), we detect it as an anomaly and increase the victim score appropriately. Similarly we measure Number of packets Out, Amplification factor out and measure the standard deviation. With this anomaly, we update the attackScore for a host, switchPort combination. Total trust Score of a host = (victimScore + 1/attackScore).
\par
\textbf{Mitigation:} If an attackScore for a host is high, rate limiting is applied to the hosts. New flows: While adding new flows to a host, we consider the victimScore of the host as well as attackScore of the source before adding a flow.




\section{\textbf{Results}}


\section{\textbf{Challeges}}
Based on the detection explained above, we further decide on how controlled a flow can be. For this, there is a parameter for each flow, priority (OFFlowmod parameter). The priority of each flow will be set based on the trust score calculated. This is primary parameter that can be used. However, the flow priority is always not that direct. If a particular flow is diagnosed as malicious it can be dropped. But if the host/switch is diagnosed malicious the action can vary. For a particular host detected malicious, all the flows matching that host will be given a lesser priority. If the switch itself is not genuine it is blocked to communicate with the controller for a while and released. This is because we don’t want to consider a good switch malicious for a suspicious activity for a single period of time. The alternative approach is QOS. There are multiple ports on a switch connected to various hosts. The traffic flow can be controlled based on the bandwidth allocated for various queues. This is one of the applications of SDN where multiple queues are configured on a switch allocated with different bandwidth. Each flow can be matched to one of these queues. The advantage of queues over priority is limiting the bandwidth of a malicious flow can reduce the risk of further damage from a flow to be very slow even if it was detected partially malicious earlier. In case of priority, if none of the other flows are active, they can dropped eventually and the malicious one still gets to operate at the complete bandwidth inspiteof being at low priority. This feature is not complete on floodlight yet as mentioned in the link above. We are trying to tune this to a level required to achieve the required functionality.
% An example of a floating figure using the graphicx package.
% Note that \label must occur AFTER (or within) \caption.
% For figures, \caption should occur after the \includegraphics.
% Note that IEEEtran v1.7 and later has special internal code that
% is designed to preserve the operation of \label within \caption
% even when the captionsoff option is in effect. However, because
% of issues like this, it may be the safest practice to put all your
% \label just after \caption rather than within \caption{}.
%
% Reminder: the "draftcls" or "draftclsnofoot", not "draft", class
% option should be used if it is desired that the figures are to be
% displayed while in draft mode.
%
%\begin{figure}[!t]
%\centering
%\includegraphics[width=2.5in]{myfigure}
% where an .eps filename suffix will be assumed under latex, 
% and a .pdf suffix will be assumed for pdflatex; or what has been declared
% via \DeclareGraphicsExtensions.
%\caption{Simulation Results}
%\label{fig_sim}
%\end{figure}

% Note that IEEE typically puts floats only at the top, even when this
% results in a large percentage of a column being occupied by floats.


% An example of a double column floating figure using two subfigures.
% (The subfig.sty package must be loaded for this to work.)
% The subfigure \label commands are set within each subfloat command, the
% \label for the overall figure must come after \caption.
% \hfil must be used as a separator to get equal spacing.
% The subfigure.sty package works much the same way, except \subfigure is
% used instead of \subfloat.
%
%\begin{figure*}[!t]
%\centerline{\subfloat[Case I]\includegraphics[width=2.5in]{subfigcase1}%
%\label{fig_first_case}}
%\hfil
%\subfloat[Case II]{\includegraphics[width=2.5in]{subfigcase2}%
%\label{fig_second_case}}}
%\caption{Simulation results}
%\label{fig_sim}
%\end{figure*}
%
% Note that often IEEE papers with subfigures do not employ subfigure
% captions (using the optional argument to \subfloat), but instead will
% reference/describe all of them (a), (b), etc., within the main caption.


% An example of a floating table. Note that, for IEEE style tables, the 
% \caption command should come BEFORE the table. Table text will default to
% \footnotesize as IEEE normally uses this smaller font for tables.
% The \label must come after \caption as always.
%
%\begin{table}[!t]
%% increase table row spacing, adjust to taste
%\renewcommand{\arraystretch}{1.3}
% if using array.sty, it might be a good idea to tweak the value of
% \extrarowheight as needed to properly center the text within the cells
%\caption{An Example of a Table}
%\label{table_example}
%\centering
%% Some packages, such as MDW tools, offer better commands for making tables
%% than the plain LaTeX2e tabular which is used here.
%\begin{tabular}{|c||c|}
%\hline
%One & Two\\
%\hline
%Three & Four\\
%\hline
%\end{tabular}
%\end{table}


% Note that IEEE does not put floats in the very first column - or typically
% anywhere on the first page for that matter. Also, in-text middle ("here")
% positioning is not used. Most IEEE journals/conferences use top floats
% exclusively. Note that, LaTeX2e, unlike IEEE journals/conferences, places
% footnotes above bottom floats. This can be corrected via the \fnbelowfloat
% command of the stfloats package.



\section{\textbf{Conclusion}}
The conclusion goes here.




% conference papers do not normally have an appendix


% use section* for acknowledgement
\section*{\textbf{Acknowledgment}}


The authors would like to thank...





% trigger a \newpage just before the given reference
% number - used to balance the columns on the last page
% adjust value as needed - may need to be readjusted if
% the document is modified later
%\IEEEtriggeratref{8}
% The "triggered" command can be changed if desired:
%\IEEEtriggercmd{\enlargethispage{-5in}}

% references section

% can use a bibliography generated by BibTeX as a .bbl file
% BibTeX documentation can be easily obtained at:
% http://www.ctan.org/tex-archive/biblio/bibtex/contrib/doc/
% The IEEEtran BibTeX style support page is at:
% http://www.michaelshell.org/tex/ieeetran/bibtex/
%\bibliographystyle{IEEEtran}
% argument is your BibTeX string definitions and bibliography database(s)
%\bibliography{IEEEabrv,../bib/paper}
%
% <OR> manually copy in the resultant .bbl file
% set second argument of \begin to the number of references
% (used to reserve space for the reference number labels box)
\begin{thebibliography}{1}

\bibitem{IEEEhowto:kopka}
H.~Kopka and P.~W. Daly, \emph{A Guide to \LaTeX}, 3rd~ed.\hskip 1em plus
  0.5em minus 0.4em\relax Harlow, England: Addison-Wesley, 1999.

\end{thebibliography}




% that's all folks
\end{document}


